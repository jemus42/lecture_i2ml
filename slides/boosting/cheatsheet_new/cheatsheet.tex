\documentclass{beamer}

\usepackage[orientation=landscape,size=a0,scale=1.4,debug]{beamerposter}
\mode<presentation>{\usetheme{mlr}}

\usepackage[utf8]{inputenc} % UTF-8
\usepackage[english]{babel} % Language
\usepackage{hyperref} % Hyperlinks
\usepackage{ragged2e} % Text position
\usepackage[export]{adjustbox} % Image position
\usepackage[most]{tcolorbox}
%\usepackage{nomencl}
%\makenomenclature
\usepackage{amsmath}
\usepackage{bm}
\usepackage{mathtools}
\usepackage{dsfont}
\usepackage{verbatim}
\usepackage{amsmath}
\usepackage{amsfonts}
\usepackage{csquotes}
\usepackage{multirow}
\usepackage{longtable}
\usepackage{enumerate}
\usepackage[absolute,overlay]{textpos}
\usepackage{psfrag}
\usepackage{algorithm}
\usepackage{algpseudocode}
\usepackage{eqnarray}
\usepackage{arydshln}
\usepackage{tabularx}
\usepackage{placeins}
\usepackage{tikz}
\usepackage{setspace}
\usepackage{colortbl}
\usepackage{mathtools}
\usepackage{wrapfig}

% math spaces
\ifdefined\N
\renewcommand{\N}{\mathds{N}} % N, naturals
\else \newcommand{\N}{\mathds{N}} \fi
\newcommand{\Z}{\mathds{Z}} % Z, integers
\newcommand{\Q}{\mathds{Q}} % Q, rationals
\newcommand{\R}{\mathds{R}} % R, reals
\ifdefined\C
  \renewcommand{\C}{\mathds{C}} % C, complex
\else \newcommand{\C}{\mathds{C}} \fi
\newcommand{\continuous}{\mathcal{C}} % C, space of continuous functions
\newcommand{\M}{\mathcal{M}} % machine numbers
\newcommand{\epsm}{\epsilon_m} % maximum error

% counting / finite sets
\newcommand{\setzo}{\{0, 1\}} % set 0, 1
\newcommand{\setmp}{\{-1, +1\}} % set -1, 1
\newcommand{\unitint}{[0, 1]} % unit interval

% basic math stuff
\newcommand{\xt}{\tilde x} % x tilde
\DeclareMathOperator*{\argmax}{arg\,max} % argmax
\DeclareMathOperator*{\argmin}{arg\,min} % argmin
\newcommand{\argminlim}{\mathop{\mathrm{arg\,min}}\limits} % argmax with limits
\newcommand{\argmaxlim}{\mathop{\mathrm{arg\,max}}\limits} % argmin with limits
\newcommand{\sign}{\operatorname{sign}} % sign, signum
\newcommand{\I}{\mathbb{I}} % I, indicator
\newcommand{\order}{\mathcal{O}} % O, order
\newcommand{\bigO}{\mathcal{O}} % Big-O Landau
\newcommand{\littleo}{{o}} % Little-o Landau
\newcommand{\pd}[2]{\frac{\partial{#1}}{\partial #2}} % partial derivative
\newcommand{\floorlr}[1]{\left\lfloor #1 \right\rfloor} % floor
\newcommand{\ceillr}[1]{\left\lceil #1 \right\rceil} % ceiling
\newcommand{\indep}{\perp \!\!\! \perp} % independence symbol

% sums and products
\newcommand{\sumin}{\sum\limits_{i=1}^n} % summation from i=1 to n
\newcommand{\sumim}{\sum\limits_{i=1}^m} % summation from i=1 to m
\newcommand{\sumjn}{\sum\limits_{j=1}^n} % summation from j=1 to p
\newcommand{\sumjp}{\sum\limits_{j=1}^p} % summation from j=1 to p
\newcommand{\sumik}{\sum\limits_{i=1}^k} % summation from i=1 to k
\newcommand{\sumkg}{\sum\limits_{k=1}^g} % summation from k=1 to g
\newcommand{\sumjg}{\sum\limits_{j=1}^g} % summation from j=1 to g
\newcommand{\meanin}{\frac{1}{n} \sum\limits_{i=1}^n} % mean from i=1 to n
\newcommand{\meanim}{\frac{1}{m} \sum\limits_{i=1}^m} % mean from i=1 to n
\newcommand{\meankg}{\frac{1}{g} \sum\limits_{k=1}^g} % mean from k=1 to g
\newcommand{\prodin}{\prod\limits_{i=1}^n} % product from i=1 to n
\newcommand{\prodkg}{\prod\limits_{k=1}^g} % product from k=1 to g
\newcommand{\prodjp}{\prod\limits_{j=1}^p} % product from j=1 to p

% linear algebra
\newcommand{\one}{\boldsymbol{1}} % 1, unitvector
\newcommand{\zero}{\mathbf{0}} % 0-vector
\newcommand{\id}{\boldsymbol{I}} % I, identity
\newcommand{\diag}{\operatorname{diag}} % diag, diagonal
\newcommand{\trace}{\operatorname{tr}} % tr, trace
\newcommand{\spn}{\operatorname{span}} % span
\newcommand{\scp}[2]{\left\langle #1, #2 \right\rangle} % <.,.>, scalarproduct
\newcommand{\mat}[1]{\begin{pmatrix} #1 \end{pmatrix}} % short pmatrix command
\newcommand{\Amat}{\mathbf{A}} % matrix A
\newcommand{\Deltab}{\mathbf{\Delta}} % error term for vectors

% basic probability + stats
\renewcommand{\P}{\mathds{P}} % P, probability
\newcommand{\E}{\mathds{E}} % E, expectation
\newcommand{\var}{\mathsf{Var}} % Var, variance
\newcommand{\cov}{\mathsf{Cov}} % Cov, covariance
\newcommand{\corr}{\mathsf{Corr}} % Corr, correlation
\newcommand{\normal}{\mathcal{N}} % N of the normal distribution
\newcommand{\iid}{\overset{i.i.d}{\sim}} % dist with i.i.d superscript
\newcommand{\distas}[1]{\overset{#1}{\sim}} % ... is distributed as ...

% machine learning
\newcommand{\Xspace}{\mathcal{X}} % X, input space
\newcommand{\Yspace}{\mathcal{Y}} % Y, output space
\newcommand{\Zspace}{\mathcal{Z}} % Z, space of sampled datapoints
\newcommand{\nset}{\{1, \ldots, n\}} % set from 1 to n
\newcommand{\pset}{\{1, \ldots, p\}} % set from 1 to p
\newcommand{\gset}{\{1, \ldots, g\}} % set from 1 to g
\newcommand{\Pxy}{\mathbb{P}_{xy}} % P_xy
\newcommand{\Exy}{\mathbb{E}_{xy}} % E_xy: Expectation over random variables xy
\newcommand{\xv}{\mathbf{x}} % vector x (bold)
\newcommand{\xtil}{\tilde{\mathbf{x}}} % vector x-tilde (bold)
\newcommand{\yv}{\mathbf{y}} % vector y (bold)
\newcommand{\xy}{(\xv, y)} % observation (x, y)
\newcommand{\xvec}{\left(x_1, \ldots, x_p\right)^\top} % (x1, ..., xp)
\newcommand{\Xmat}{\mathbf{X}} % Design matrix
\newcommand{\allDatasets}{\mathds{D}} % The set of all datasets
\newcommand{\allDatasetsn}{\mathds{D}_n}  % The set of all datasets of size n
\newcommand{\D}{\mathcal{D}} % D, data
\newcommand{\Dn}{\D_n} % D_n, data of size n
\newcommand{\Dtrain}{\mathcal{D}_{\text{train}}} % D_train, training set
\newcommand{\Dtest}{\mathcal{D}_{\text{test}}} % D_test, test set
\newcommand{\xyi}[1][i]{\left(\xv^{(#1)}, y^{(#1)}\right)} % (x^i, y^i), i-th observation
\newcommand{\Dset}{\left( \xyi[1], \ldots, \xyi[n]\right)} % {(x1,y1)), ..., (xn,yn)}, data
\newcommand{\defAllDatasetsn}{(\Xspace \times \Yspace)^n} % Def. of the set of all datasets of size n
\newcommand{\defAllDatasets}{\bigcup_{n \in \N}(\Xspace \times \Yspace)^n} % Def. of the set of all datasets
\newcommand{\xdat}{\left\{ \xv^{(1)}, \ldots, \xv^{(n)}\right\}} % {x1, ..., xn}, input data
\newcommand{\ydat}{\left\{ \yv^{(1)}, \ldots, \yv^{(n)}\right\}} % {y1, ..., yn}, input data
\newcommand{\yvec}{\left(y^{(1)}, \hdots, y^{(n)}\right)^\top} % (y1, ..., yn), vector of outcomes
\newcommand{\greekxi}{\xi} % Greek letter xi
\renewcommand{\xi}[1][i]{\xv^{(#1)}} % x^i, i-th observed value of x
\newcommand{\yi}[1][i]{y^{(#1)}} % y^i, i-th observed value of y
\newcommand{\xivec}{\left(x^{(i)}_1, \ldots, x^{(i)}_p\right)^\top} % (x1^i, ..., xp^i), i-th observation vector
\newcommand{\xj}{\xv_j} % x_j, j-th feature
\newcommand{\xjvec}{\left(x^{(1)}_j, \ldots, x^{(n)}_j\right)^\top} % (x^1_j, ..., x^n_j), j-th feature vector
\newcommand{\phiv}{\mathbf{\phi}} % Basis transformation function phi
\newcommand{\phixi}{\mathbf{\phi}^{(i)}} % Basis transformation of xi: phi^i := phi(xi)

%%%%%% ml - models general
\newcommand{\lamv}{\bm{\lambda}} % lambda vector, hyperconfiguration vector
\newcommand{\Lam}{\bm{\Lambda}}	 % Lambda, space of all hpos
% Inducer / Inducing algorithm
\newcommand{\preimageInducer}{\left(\defAllDatasets\right)\times\Lam} % Set of all datasets times the hyperparameter space
\newcommand{\preimageInducerShort}{\allDatasets\times\Lam} % Set of all datasets times the hyperparameter space
% Inducer / Inducing algorithm
\newcommand{\ind}{\mathcal{I}} % Inducer, inducing algorithm, learning algorithm

% continuous prediction function f
\newcommand{\ftrue}{f_{\text{true}}}  % True underlying function (if a statistical model is assumed)
\newcommand{\ftruex}{\ftrue(\xv)} % True underlying function (if a statistical model is assumed)
\newcommand{\fx}{f(\xv)} % f(x), continuous prediction function
\newcommand{\fdomains}{f: \Xspace \rightarrow \R^g} % f with domain and co-domain
\newcommand{\Hspace}{\mathcal{H}} % hypothesis space where f is from
\newcommand{\fbayes}{f^{\ast}} % Bayes-optimal model
\newcommand{\fxbayes}{f^{\ast}(\xv)} % Bayes-optimal model
\newcommand{\fkx}[1][k]{f_{#1}(\xv)} % f_j(x), discriminant component function
\newcommand{\fh}{\hat{f}} % f hat, estimated prediction function
\newcommand{\fxh}{\fh(\xv)} % fhat(x)
\newcommand{\fxt}{f(\xv ~|~ \thetav)} % f(x | theta)
\newcommand{\fxi}{f\left(\xv^{(i)}\right)} % f(x^(i))
\newcommand{\fxih}{\hat{f}\left(\xv^{(i)}\right)} % f(x^(i))
\newcommand{\fxit}{f\left(\xv^{(i)} ~|~ \thetav\right)} % f(x^(i) | theta)
\newcommand{\fhD}{\fh_{\D}} % fhat_D, estimate of f based on D
\newcommand{\fhDtrain}{\fh_{\Dtrain}} % fhat_Dtrain, estimate of f based on D
\newcommand{\fhDnlam}{\fh_{\Dn, \lamv}} %model learned on Dn with hp lambda
\newcommand{\fhDlam}{\fh_{\D, \lamv}} %model learned on D with hp lambda
\newcommand{\fhDnlams}{\fh_{\Dn, \lamv^\ast}} %model learned on Dn with optimal hp lambda
\newcommand{\fhDlams}{\fh_{\D, \lamv^\ast}} %model learned on D with optimal hp lambda

% discrete prediction function h
\newcommand{\hx}{h(\xv)} % h(x), discrete prediction function
\newcommand{\hh}{\hat{h}} % h hat
\newcommand{\hxh}{\hat{h}(\xv)} % hhat(x)
\newcommand{\hxt}{h(\xv | \thetav)} % h(x | theta)
\newcommand{\hxi}{h\left(\xi\right)} % h(x^(i))
\newcommand{\hxit}{h\left(\xi ~|~ \thetav\right)} % h(x^(i) | theta)
\newcommand{\hbayes}{h^{\ast}} % Bayes-optimal classification model
\newcommand{\hxbayes}{h^{\ast}(\xv)} % Bayes-optimal classification model

% yhat
\newcommand{\yh}{\hat{y}} % yhat for prediction of target
\newcommand{\yih}{\hat{y}^{(i)}} % yhat^(i) for prediction of ith targiet
\newcommand{\resi}{\yi- \yih}

% theta
\newcommand{\thetah}{\hat{\theta}} % theta hat
\newcommand{\thetav}{\bm{\theta}} % theta vector
\newcommand{\thetavh}{\bm{\hat\theta}} % theta vector hat
\newcommand{\thetat}[1][t]{\thetav^{[#1]}} % theta^[t] in optimization
\newcommand{\thetatn}[1][t]{\thetav^{[#1 +1]}} % theta^[t+1] in optimization
\newcommand{\thetahDnlam}{\thetavh_{\Dn, \lamv}} %theta learned on Dn with hp lambda
\newcommand{\thetahDlam}{\thetavh_{\D, \lamv}} %theta learned on D with hp lambda
\newcommand{\mint}{\min_{\thetav \in \Theta}} % min problem theta
\newcommand{\argmint}{\argmin_{\thetav \in \Theta}} % argmin theta

% densities + probabilities
% pdf of x
\newcommand{\pdf}{p} % p
\newcommand{\pdfx}{p(\xv)} % p(x)
\newcommand{\pixt}{\pi(\xv~|~ \thetav)} % pi(x|theta), pdf of x given theta
\newcommand{\pixit}[1][i]{\pi\left(\xi[#1] ~|~ \thetav\right)} % pi(x^i|theta), pdf of x given theta
\newcommand{\pixii}[1][i]{\pi\left(\xi[#1]\right)} % pi(x^i), pdf of i-th x

% pdf of (x, y)
\newcommand{\pdfxy}{p(\xv,y)} % p(x, y)
\newcommand{\pdfxyt}{p(\xv, y ~|~ \thetav)} % p(x, y | theta)
\newcommand{\pdfxyit}{p\left(\xi, \yi ~|~ \thetav\right)} % p(x^(i), y^(i) | theta)

% pdf of x given y
\newcommand{\pdfxyk}[1][k]{p(\xv | y= #1)} % p(x | y = k)
\newcommand{\lpdfxyk}[1][k]{\log p(\xv | y= #1)} % log p(x | y = k)
\newcommand{\pdfxiyk}[1][k]{p\left(\xi | y= #1 \right)} % p(x^i | y = k)

% prior probabilities
\newcommand{\pik}[1][k]{\pi_{#1}} % pi_k, prior
\newcommand{\lpik}[1][k]{\log \pi_{#1}} % log pi_k, log of the prior
\newcommand{\pit}{\pi(\thetav)} % Prior probability of parameter theta

% posterior probabilities
\newcommand{\post}{\P(y = 1 ~|~ \xv)} % P(y = 1 | x), post. prob for y=1
\newcommand{\postk}[1][k]{\P(y = #1 ~|~ \xv)} % P(y = k | y), post. prob for y=k
\newcommand{\pidomains}{\pi: \Xspace \rightarrow \unitint} % pi with domain and co-domain
\newcommand{\pibayes}{\pi^{\ast}} % Bayes-optimal classification model
\newcommand{\pixbayes}{\pi^{\ast}(\xv)} % Bayes-optimal classification model
\newcommand{\pix}{\pi(\xv)} % pi(x), P(y = 1 | x)
\newcommand{\piv}{\bm{\pi}} % pi, bold, as vector
\newcommand{\pikx}[1][k]{\pi_{#1}(\xv)} % pi_k(x), P(y = k | x)
\newcommand{\pikxt}[1][k]{\pi_{#1}(\xv ~|~ \thetav)} % pi_k(x | theta), P(y = k | x, theta)
\newcommand{\pixh}{\hat \pi(\xv)} % pi(x) hat, P(y = 1 | x) hat
\newcommand{\pikxh}[1][k]{\hat \pi_{#1}(\xv)} % pi_k(x) hat, P(y = k | x) hat
\newcommand{\pixih}{\hat \pi(\xi)} % pi(x^(i)) with hat
\newcommand{\pikxih}[1][k]{\hat \pi_{#1}(\xi)} % pi_k(x^(i)) with hat
\newcommand{\pdfygxt}{p(y ~|~\xv, \thetav)} % p(y | x, theta)
\newcommand{\pdfyigxit}{p\left(\yi ~|~\xi, \thetav\right)} % p(y^i |x^i, theta)
\newcommand{\lpdfygxt}{\log \pdfygxt } % log p(y | x, theta)
\newcommand{\lpdfyigxit}{\log \pdfyigxit} % log p(y^i |x^i, theta)

% probababilistic
\newcommand{\bayesrulek}[1][k]{\frac{\P(\xv | y= #1) \P(y= #1)}{\P(\xv)}} % Bayes rule
\newcommand{\muk}{\bm{\mu_k}} % mean vector of class-k Gaussian (discr analysis)

% residual and margin
\newcommand{\eps}{\epsilon} % residual, stochastic
\newcommand{\epsv}{\bm{\epsilon}} % residual, stochastic, as vector
\newcommand{\epsi}{\epsilon^{(i)}} % epsilon^i, residual, stochastic
\newcommand{\epsh}{\hat{\epsilon}} % residual, estimated
\newcommand{\epsvh}{\hat{\epsv}} % residual, estimated, vector
\newcommand{\yf}{y \fx} % y f(x), margin
\newcommand{\yfi}{\yi \fxi} % y^i f(x^i), margin
\newcommand{\Sigmah}{\hat \Sigma} % estimated covariance matrix
\newcommand{\Sigmahj}{\hat \Sigma_j} % estimated covariance matrix for the j-th class

% ml - loss, risk, likelihood
\newcommand{\Lyf}{L\left(y, f\right)} % L(y, f), loss function
\newcommand{\Lypi}{L\left(y, \pi\right)} % L(y, pi), loss function
\newcommand{\Lxy}{L\left(y, \fx\right)} % L(y, f(x)), loss function
\newcommand{\Lxyi}{L\left(\yi, \fxi\right)} % loss of observation
\newcommand{\Lxyt}{L\left(y, \fxt\right)} % loss with f parameterized
\newcommand{\Lxyit}{L\left(\yi, \fxit\right)} % loss of observation with f parameterized
\newcommand{\Lxym}{L\left(\yi, f\left(\bm{\tilde{x}}^{(i)} ~|~ \thetav\right)\right)} % loss of observation with f parameterized
\newcommand{\Lpixy}{L\left(y, \pix\right)} % loss in classification
\newcommand{\Lpiv}{L\left(y, \piv\right)} % loss in classification
\newcommand{\Lpixyi}{L\left(\yi, \pixii\right)} % loss of observation in classification
\newcommand{\Lpixyt}{L\left(y, \pixt\right)} % loss with pi parameterized
\newcommand{\Lpixyit}{L\left(\yi, \pixit\right)} % loss of observation with pi parameterized
\newcommand{\Lhxy}{L\left(y, \hx\right)} % L(y, h(x)), loss function on discrete classes
\newcommand{\Lr}{L\left(r\right)} % L(r), loss defined on residual (reg) / margin (classif)
\newcommand{\lone}{|y - \fx|} % L1 loss
\newcommand{\ltwo}{\left(y - \fx\right)^2} % L2 loss
\newcommand{\lbernoullimp}{\ln(1 + \exp(-y \cdot \fx))} % Bernoulli loss for -1, +1 encoding
\newcommand{\lbernoullizo}{- y \cdot \fx + \log(1 + \exp(\fx))} % Bernoulli loss for 0, 1 encoding
\newcommand{\lcrossent}{- y \log \left(\pix\right) - (1 - y) \log \left(1 - \pix\right)} % cross-entropy loss
\newcommand{\lbrier}{\left(\pix - y \right)^2} % Brier score
\newcommand{\risk}{\mathcal{R}} % R, risk
\newcommand{\riskbayes}{\mathcal{R}^\ast}
\newcommand{\riskf}{\risk(f)} % R(f), risk
\newcommand{\riskdef}{\E_{y|\xv}\left(\Lxy \right)} % risk def (expected loss)
\newcommand{\riskt}{\mathcal{R}(\thetav)} % R(theta), risk
\newcommand{\riske}{\mathcal{R}_{\text{emp}}} % R_emp, empirical risk w/o factor 1 / n
\newcommand{\riskeb}{\bar{\mathcal{R}}_{\text{emp}}} % R_emp, empirical risk w/ factor 1 / n
\newcommand{\riskef}{\riske(f)} % R_emp(f)
\newcommand{\risket}{\mathcal{R}_{\text{emp}}(\thetav)} % R_emp(theta)
\newcommand{\riskr}{\mathcal{R}_{\text{reg}}} % R_reg, regularized risk
\newcommand{\riskrt}{\mathcal{R}_{\text{reg}}(\thetav)} % R_reg(theta)
\newcommand{\riskrf}{\riskr(f)} % R_reg(f)
\newcommand{\riskrth}{\hat{\mathcal{R}}_{\text{reg}}(\thetav)} % hat R_reg(theta)
\newcommand{\risketh}{\hat{\mathcal{R}}_{\text{emp}}(\thetav)} % hat R_emp(theta)
\newcommand{\LL}{\mathcal{L}} % L, likelihood
\newcommand{\LLt}{\mathcal{L}(\thetav)} % L(theta), likelihood
\newcommand{\LLtx}{\mathcal{L}(\thetav | \xv)} % L(theta|x), likelihood
\newcommand{\logl}{\ell} % l, log-likelihood
\newcommand{\loglt}{\logl(\thetav)} % l(theta), log-likelihood
\newcommand{\logltx}{\logl(\thetav | \xv)} % l(theta|x), log-likelihood
\newcommand{\errtrain}{\text{err}_{\text{train}}} % training error
\newcommand{\errtest}{\text{err}_{\text{test}}} % test error
\newcommand{\errexp}{\overline{\text{err}_{\text{test}}}} % avg training error

% lm
\newcommand{\thx}{\thetav^\top \xv} % linear model
\newcommand{\olsest}{(\Xmat^\top \Xmat)^{-1} \Xmat^\top \yv} % OLS estimator in LM

% ml - boosting
\newcommand{\fm}{f^{[m]}}                                                   % prediction in iteration m
\newcommand{\fmh}{\hat{f}^{[m]}}                                            % prediction in iteration m
\newcommand{\fmd}{f^{[m-1]}}                                                % prediction m-1
\newcommand{\fmdh}{\hat{f}^{[m-1]}}                                         % prediction m-1
\newcommand{\bmm}{b^{[m]}}                                                  % basemodel m
\newcommand{\bmmh}{\hat{b}^{[m]}}                                           % basemodel m with hat
\newcommand{\betam}{\beta^{[m]}}                                            % weight of basemodel m
\newcommand{\betamh}{\hat{\beta}^{[m]}}                                     % weight of basemodel m with hat
\newcommand{\betai}[1]{\beta^{[#1]}}                                        % weight of basemodel with argument for m
\newcommand{\errm}{\text{err}^{[m]}}                                        % weighted in-sample misclassification rate
\newcommand{\wm}{w^{[m]}}                                                   % weight vector of basemodel m
\newcommand{\wmi}{w^{[m](i)}}                                               % weight of obs i of basemodel m
\newcommand{\thetam}{\theta^{[m]}}                                          % parameters of basemodel m
\newcommand{\thetamh}{\hat{\theta}^{[m]}}                                   % parameters of basemodel m with hat
\newcommand{\rmm}{r^{[m]}}                                                  % pseudo residuals
\newcommand{\rmi}{r^{[m](i)}}                                               % pseudo residuals
\newcommand{\Rtm}{R_{t}^{[m]}}                                              % terminal-region
\newcommand{\Tm}{T^{[m]}}                                                   %
\newcommand{\ctm}{c_t^{[m]}}                                                % mean, terminal-regions
\newcommand{\ctmh}{\hat{c}_t^{[m]}}                                         % mean, terminal-regions with hat
\newcommand{\ctmt}{\tilde{c}_t^{[m]}}                                       % mean, terminal-regions
\newcommand{\fxk}{f_k(x)}                                                   % f_k(x)
\newcommand{\Lp}{L^\prime}
\newcommand{\Ldp}{L^{\prime\prime}}
\newcommand{\Lpleft}{\Lp_{\text{left}}}
{}

\title{CIM2 - Boosting :\,: CHEAT SHEET} % Package title in header, \, adds thin space between ::
\newcommand{\packagedescription}{ % Package description in header
	The \textbf{I2ML}: Introduction to ML course offers an introductory and applied overview of "supervised" ML. It is organized as a digital lecture.
}

\newlength{\columnheight} % Adjust depending on header height
\setlength{\columnheight}{84cm}

\newtcolorbox{codebox}{%
	sharp corners,
	leftrule=0pt,
	rightrule=0pt,
	toprule=0pt,
	bottomrule=0pt,
	hbox}

\newtcolorbox{codeboxmultiline}[1][]{%
	sharp corners,
	leftrule=0pt,
	rightrule=0pt,
	toprule=0pt,
	bottomrule=0pt,
	#1}

\begin{document}
\begin{frame}[fragile]{}
\begin{columns}
	\begin{column}{.31\textwidth}
		\begin{beamercolorbox}[center]{postercolumn}
			\begin{minipage}{.98\textwidth}
				\parbox[t][\columnheight]{\textwidth}{
					\vspace{1cm}
					\textbf{Note: } The lectures covers a wide range of topics and uses a variety of examples from different areas within mathematics, statistics and ML. Therefore notation may overlap. The context should make clear how to uncerstand the notation. % For example, $L$ denotes the likelihood function $L$ in a statistical context, while it denotes the loss function $L$ in ML context).  \\
					If you notice ambiguities that do not clarify if the context is taken into account, please contact the instructors.
					\begin{myblock}{Boosting - General}
						\begin{codebox}
							$\mathcal{B}$ : Hypothesis space of base learners
						\end{codebox}
						\hspace*{1ex}
						\begin{codebox}
						    $\thetam$ : Model parameter vector of $m$-th iteration
						\end{codebox}
						\hspace*{1ex}
						\begin{codebox}
							 $\bmm$ or $\bmm (\xv)$ or $\bmmxth \in \mathcal{B}$: Base learner of iteration $m$
						\end{codebox}
						\hspace*{1ex}
						\begin{codebox}
						    $\betam$ :  Weights in AdaBoost algorithm of iteration m
						\end{codebox}
						\hspace*{1ex}
						\begin{codebox}
							$\beta$ : Learning rate / step size in gradient boosting
						\end{codebox}
						\hspace*{1ex} 
						\begin{codebox}
						    $\thetamh , \bmmh , \betamh$
						\end{codebox}
						\hspace*{1ex} The hat symbol denotes \textbf{learned} parameters and functions\\
					\end{myblock}
			
						\begin{myblock}{Forward Stagewise Additive Modelling}
						\begin{codebox}
							 $\fx = \sum_{m=1}^M \betam \bmmxth$ : Additive model of base learners
						\end{codebox}
						\hspace*{1ex}  
						\begin{codebox}
							 $(\betamh, \thetamh) = \argmin \limits_{\beta, \bm{\theta}} \sum \limits_{i=1}^n
                 L\left(\yi, \fmdh\left(\xi\right) + \beta b\left(\xi, \bm{\theta}\right)\right)$ 
						\end{codebox}
						\hspace*{1ex} Empirical risk is minimized sequentially w.r.t the next additive component \\
						\begin{codebox}
							 $\fmh(\xv) = \fmdh(\xv) + \betamh b\left(\xv, \thetamh\right)$ 
						\end{codebox}
						\hspace*{1ex} Update of learned additive model \\
					\end{myblock}					
					\vfill
					}
			\end{minipage}
		\end{beamercolorbox}
	\end{column}
	\begin{column}{.31\textwidth}
		\begin{beamercolorbox}[center]{postercolumn}
			\begin{minipage}{.98\textwidth}
				\parbox[t][\columnheight]{\textwidth}{
				\begin{myblock}{Gradient Boosting - Regression}
						\begin{codebox}
							 $\hat{f}^{[0]}(\xv) = \argmin_{\bm{\theta}} \sumin L(\yi, b(\xi, \bm{\theta}))$ 
						\end{codebox}
						\hspace*{1ex}Initialize by the optimal constant value depending on the loss function\\
						\begin{codebox}
							 $\rmi = -\left[\fp{\Lxyi}{\fxi}\right]_{f=\fmdh}$
						\end{codebox}
						\hspace*{1ex} Pseudo-residuals (negative gradient) of iteration $m$ and observation $i$\\
							\begin{codebox}
							 $\thetamh = \argmin \limits_{\bm{\theta}} \sumin (\rmi - b(\xi, \bm{\theta}))^2$
						\end{codebox}
						\hspace*{1ex} Fit a regression base learner to the pseudo-residuals to determine new parameter estimates.\\
						\begin{codebox}
							 $\fmh(\xv) = \fmdh(\xv) + \beta b(\xv, \thetamh)$
						\end{codebox}
						\hspace*{1ex} Update step of boosting model by adding fitted base model of current iteration multiplied by constant learning rate to $\fmdh(\xv)$. \\
						\end{myblock}
					\begin{myblock}{Gradient Boosting with Trees}
						\begin{codebox}
							$R_t^{[m]}$ : Terminal region $t$ of iteration $m$
						\end{codebox}
						\hspace*{1ex}
						\begin{codebox}
							 $\hat{c}_t^{[m]}$ : Loss optimal constant parameter of region $t$ and iteration $m$
						\end{codebox}
						\hspace*{1ex} 
						\begin{codebox}
							 $\bmmh(\xv) = \sum_{t=1}^{T} \ctmh \mathds{1}_{\{x \in R_t\}}$ : Tree base learner in iteration $m$
						\end{codebox}
						\hspace*{1ex}
						\begin{codebox}
							 $\fmh(\xv) = \fmdh(\xv) + \bmmh(\xv)$ : Model parameter and its estimate
						\end{codebox}
						\hspace*{1ex} Update step of boosting model by adding fitted base model of current iteration. Note that the learning rate has been included in $\hat{c}_t^{[m]}$ since we assume a constant learning rate. \\
					\end{myblock}					
				}
			\end{minipage}
		\end{beamercolorbox}
	\end{column}


	\begin{column}{.31\textwidth}
		\begin{beamercolorbox}[center]{postercolumn}
			\begin{minipage}{.98\textwidth}
				\parbox[t][\columnheight]{\textwidth}{
				\begin{myblock}{Gradient Boosting - Classification}
						
					\end{myblock}					
				}
			\end{minipage}
		\end{beamercolorbox}
	\end{column}
\end{columns}


\end{frame}

% \newpage
% 
% \begin{frame}[fragile]{}
% \begin{columns}
% 	\begin{column}{.31\textwidth}
% 		\begin{beamercolorbox}[center]{postercolumn}
% 			\begin{minipage}{.98\textwidth}
% 				\parbox[t][\columnheight]{\textwidth}{
% 					\vspace{1cm}
% 					\begin{myblock}{Numerical Analysis}
% 						\begin{codebox}
% 							$\xv, \tilde \xv \in \R^n$ : Un-disturbed / disturbed input
% 						\end{codebox}
% 						\hspace*{1ex}
% 						\begin{codebox}
% 							$\Delta \xv, \delta \xv$ : Absolute / relative error in the computation of $\xv$
% 						\end{codebox}
% 						\hspace*{1ex}
% 						\begin{codebox}
% 						$\kappa$ : Smallest $\kappa \geq 0$, so that
% 						$\frac{|f(x+\Delta x)-y|}{|y|} \leq \kappa \ \frac{|\Delta x|}{|x|} \text{ for } \Delta x \to 0.$
% 						\end{codebox}
% 						\hspace*{1ex} $\kappa$ is called condition number.\\
% 						\begin{codebox}
% 							$f, \tilde f: \R^n \to \R^m$ : A problem $f$ and an algorithm $\tilde f$ to solve $f$
% 						\end{codebox}
% 					\end{myblock}
% 					\begin{myblock}{Big-O}
% 						\textbf{Note}: In this chapter, all functions are real-valued functions.
% 						\vspace*{3ex}
% 						\begin{codebox}
% 							$f \in \order(g) \text{ for } x \to \infty$ : $\lim\limits_{x \to \infty} \sup \frac{f(x)}{g(x)} < \infty$
% 						\end{codebox}
% 						\hspace*{1ex} \enquote{Big-O-Notation}. $f$ runs in the order of $g$. \\
% 						% \begin{codebox}
% 						% 	$f \in o(g) \text{ for } x \to \infty$ : $\lim\limits_{x \to \infty} \frac{f(x)}{g(x)} = 0$
% 						% \end{codebox}
% 						% \hspace*{1ex} \enquote{Little-O-Notation}. $g$ grows much faster than $f$. \\
% 						\begin{codebox}
% 						$f \in \order(1)\}$ : $f$ has constant runtime complexity
% 						\end{codebox}
% 						\hspace*{1ex}
% 						\begin{codebox}
% 						$f \in \order(\log(n)) $ : $f$ has logarithmic runtime complexity
% 						\end{codebox}
% 						\hspace*{1ex}
% 						\begin{codebox}
% 						$f \in \order(n^c) $ : $f$ has polynomial runtime complexity
% 						\end{codebox}
% 						\hspace*{1ex} It is called linear for $c = 1$, quadratic for $c = 2$ and cubic for $c = 3$. \\
% 						\begin{codebox}
% 						$f \in \order(c^n) $ : $f$ has exponential runtime complexity
% 						\end{codebox}
% 					\end{myblock}
% 					\begin{myblock}{Quadrature}
% 						\textbf{Note}: In this chapter, all functions are real-valued functions.
% 						\vspace*{3ex}
% 						\begin{codebox}
% 							$I(f) = \int_a^b f(x)~dx$ : (Riemann) integral of $f$
% 						\end{codebox}
% 						\hspace*{1ex}
% 						\begin{codebox}
% 							$Q(f)$ : Numerical approximation of $I(f)$ based on a quadrature rule $Q$
% 						\end{codebox}
% 						\hspace*{1ex}
% 						\begin{codebox}
% 							$|\Delta I(f)| = |Q(f) - I(f)|$ : Error of the numerical approximation
% 						\end{codebox}
% 					\end{myblock}		
% 					% \begin{myblock}{Matrix Decomposition}
% 					% 	\begin{codebox}
% 					% 		$\sum_{k = 1}^n k = \frac{n(n+1)}{2}$ : Sum of natural numbers by Carl Friedrich Gauss
% 					% 	\end{codebox}
% 					% 	\hspace*{1ex}
% 					% 	\begin{codebox}
% 					% 		$\sum_{k = 1}^n k^2 = \frac{n(n+1)(2n + 1)}{6}$ : Sum of squares of first n numbers
% 					% 	\end{codebox}
% 					% \end{myblock}													
% 					\vfill
% 				}
% 			\end{minipage}
% 		\end{beamercolorbox}
% 	\end{column}
% 	\begin{column}{.31\textwidth}
% 		\begin{beamercolorbox}[center]{postercolumn}
% 			\begin{minipage}{.98\textwidth}
% 				\parbox[t][\columnheight]{\textwidth}{
% 					\vspace*{0.75cm}
% 					\begin{myblock}{Optimization (General)}
% 						\begin{codebox}
% 							$\min\limits_{\xv \in \mathcal{S}} f(\xv)$ : Optimization Problem
% 						\end{codebox}
% 						\hspace*{1ex} General formulation of a (constrained) optimization problem. \\
% 						\begin{codebox}
% 							$\xv \in \mathcal{S}$: Decision variable $\xv$ in the decision space $\mathcal{S}$. 
% 						\end{codebox}
% 						\begin{codebox}
% 							 $f: \mathcal{S} \to \R^m$ : Function with domain $\mathcal{S} \subseteq \R^d$ and codomain $\R^m$
% 						\end{codebox}
% 						\hspace*{1ex} 
% 						\begin{codebox}
% 							$\min\limits_{\thetab \in \Theta} \risket$ : Empirical Risk Minimization (ERM) Problem
% 						\end{codebox}
% 						\hspace*{1ex} \textbf{Note:} We often consider the ERM problem in ML as an example of an optimization problem. In this case, we switch from the general optimization notation to the ML notation: 
% 						\begin{itemize}
% 							\item We optimize the function $\risket$ (instead of $f$); $f$ instead denotes the ML model. 
% 							\item We optimize over $\thetab \in \Theta$ (instead over $\xv \in \mathcal{S}$). 
% 						\end{itemize}
% 						All further notation changes accordingly. \\
% 						\hspace*{1ex}			
% 						\begin{codebox}
% 							 $\xv^\ast \in \argmin\limits_{\xv \in \mathcal{S}} f(x)$, $y^\ast = \min\limits_{\xv \in \mathcal{S}} f(x)$
% 						\end{codebox}
% 						\hspace*{1ex} Theoretical optimum of a function $f$; $y^\ast = f(\xv^\ast)$. \\
% 						\begin{codebox}
% 							 $\hat\xv \in \mathcal{S}, \hat y \in \R$ : Estimation for the optimal point and optimal value
% 						\end{codebox}
% 						\hspace*{1ex} Returned output of an optimizer; $(\hat\xv, \hat y) = \mathcal{A}(f,\mathcal{S})$ with algorithm $\mathcal{A}$. \\
% 						\begin{codebox}
% 							 $\xv^{[t]} \in \mathcal{S}$ : $t$-th step of an optimizer in the decision space
% 						\end{codebox}
% 						\vspace*{2ex} \textbf{Multivariate Optimization}\\
% 							\begin{codebox}
% 								 $\alpha \in \R_+$ : Step-size / learning rate
% 							\end{codebox}
% 							\hspace*{1ex}
% 							\begin{codebox}
% 								 $\bm{d} \in \R^d$ : Descent direction in $\xv$
% 							\end{codebox}
% 							\hspace*{1ex}
% 							% \begin{codebox}
% 							% 	 $\bm{\nu} \in \R^d$ : Velocity
% 							% \end{codebox}
% 							% \hspace*{1ex}
% 							\begin{codebox}
% 								 $\varphi \in [0, 1]$ : Momentum
% 							\end{codebox}
% 						\end{myblock}	
% 					\begin{myblock}{Optimization (Constrained Optimization)}
% 					\begin{codebox}
% 							$\min\limits_{\xv \in \mathcal{S}} f(\xv)$ s.t. $h(\xv) = 0, g(\xv) \le 0$ : Constrained Optimization Problem
% 						\end{codebox}
% 						\begin{itemize}
% 							\item \quad $h: \mathcal{S} \to \R^k$ : equality constraints; $k$: number of constraints\\
% 							\item \quad $g: \mathcal{S} \to \R^l$ : inequality constraints; $l$: number of constraints
% 						\end{itemize} \hspace{1ex} 
% 						\begin{codebox}
% 							 $\mathcal{L}: \mathcal{S} \times \R^k \times \R^l$ : Lagrangian
% 						\end{codebox}
% 						\hspace*{1ex} $(\xv, \bm{\alpha}, \bm{\beta}) \mapsto \mathcal{L}(\xv, \bm{\alpha}, \bm{\beta})$; $\bm{\alpha}, \bm{\beta}$ are called Lagrange multiplier. \\
% 					\end{myblock}											
% 					}
% 			\end{minipage}
% 		\end{beamercolorbox}
% 	\end{column}
% 	\begin{column}{.31\textwidth}
% 		\begin{beamercolorbox}[center]{postercolumn}
% 			\begin{minipage}{.98\textwidth}
% 				\parbox[t][\columnheight]{\textwidth}{
% 					\begin{myblock}{Optimization (Evolutionary Algorithms)}
% 						\begin{codebox}
% 							 $P$ : Population (of solution candidates)
% 						\end{codebox}
% 						\hspace*{1ex} 
% 						\begin{codebox}
% 							 $\mu \in \N$ : Size of a population
% 						\end{codebox}
% 						\hspace*{1ex} 
% 						\begin{codebox}
% 							 $\lambda \in \N$ : Offspring size
% 						\end{codebox}
% 						\hspace*{1ex} 
% 						\begin{codebox}
% 							 $(\mu, \lambda)$-selection : Survival selection strategy
% 						\end{codebox}
% 						\hspace*{1ex} The best $\mu$ individuals from $\lambda$ candidates are chosen ($\lambda \ge \mu$ required). \\
% 						\begin{codebox}
% 							 $(\mu + \lambda)$-selection : Survival selection strategy
% 						\end{codebox}
% 						\hspace*{1ex} The best $\mu$ individuals are chosen from the pool of the current population of size $\mu$ and the offspring of size $\lambda$. \\
% 						\begin{codebox}
% 							 $\xv_{i:\lambda}$ : $i$-th ranked candidate
% 						\end{codebox}
% 						\hspace*{1ex} $\lambda$ solution candidates are ranked according to some criterion (e.g. by a fitness function); $\xv_{i:\lambda}$ means that this candidate has rank $i$. \\
% 						\begin{codebox}
% 							 $\bm{m}^{[g]}, \bm{C}^{[g]}, \sigma^{[g]}$ : configurations in generation $g$
% 						\end{codebox}
% 						\hspace*{1ex} The superscript $[g]$ denotes the $g$-th generation. \\
% 						\begin{codebox}
% 							 $\xv^{[g](k)}$ : $k$-th individual in the population in generation $g$.
% 						\end{codebox}
% 						\end{myblock}
% 						\begin{myblock}{Optimization (Multi-Objective)}
% 						\begin{codebox}
% 							 $\mathcal{P}$ : Pareto set
% 						\end{codebox}
% 						\hspace*{1ex} Set of nondominated solutions (in the decision space $\mathcal{S}$) \\
% 						\begin{codebox}
% 							 $\mathcal{F}$ : Pareto front
% 						\end{codebox}
% 						\hspace*{1ex} Image of the Pareto set $\mathcal{P}$ under a multi-objective function $f$ \\
% 					\end{myblock}
% 
% 				}
% 			\end{minipage}
% 		\end{beamercolorbox}
% 	\end{column}
% \end{columns}
% \end{frame}

\end{document}