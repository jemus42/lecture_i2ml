
\usepackage[]{graphicx}\usepackage[]{color}
% maxwidth is the original width if it is less than linewidth
% otherwise use linewidth (to make sure the graphics do not exceed the margin)
\makeatletter
\def\maxwidth{ %
  \ifdim\Gin@nat@width>\linewidth
    \linewidth
  \else
    \Gin@nat@width
  \fi
}
\makeatother

\definecolor{fgcolor}{rgb}{0.345, 0.345, 0.345}
\newcommand{\hlnum}[1]{\textcolor[rgb]{0.686,0.059,0.569}{#1}}%
\newcommand{\hlstr}[1]{\textcolor[rgb]{0.192,0.494,0.8}{#1}}%
\newcommand{\hlcom}[1]{\textcolor[rgb]{0.678,0.584,0.686}{\textit{#1}}}%
\newcommand{\hlopt}[1]{\textcolor[rgb]{0,0,0}{#1}}%
\newcommand{\hlstd}[1]{\textcolor[rgb]{0.345,0.345,0.345}{#1}}%
\newcommand{\hlkwa}[1]{\textcolor[rgb]{0.161,0.373,0.58}{\textbf{#1}}}%
\newcommand{\hlkwb}[1]{\textcolor[rgb]{0.69,0.353,0.396}{#1}}%
\newcommand{\hlkwc}[1]{\textcolor[rgb]{0.333,0.667,0.333}{#1}}%
\newcommand{\hlkwd}[1]{\textcolor[rgb]{0.737,0.353,0.396}{\textbf{#1}}}%
\let\hlipl\hlkwb

\usepackage{framed}
\makeatletter
\newenvironment{kframe}{%
 \def\at@end@of@kframe{}%
 \ifinner\ifhmode%
  \def\at@end@of@kframe{\end{minipage}}%
  \begin{minipage}{\columnwidth}%
 \fi\fi%
 \def\FrameCommand##1{\hskip\@totalleftmargin \hskip-\fboxsep
 \colorbox{shadecolor}{##1}\hskip-\fboxsep
     % There is no \\@totalrightmargin, so:
     \hskip-\linewidth \hskip-\@totalleftmargin \hskip\columnwidth}%
 \MakeFramed {\advance\hsize-\width
   \@totalleftmargin\z@ \linewidth\hsize
   \@setminipage}}%
 {\par\unskip\endMakeFramed%
 \at@end@of@kframe}
\makeatother

\definecolor{shadecolor}{rgb}{.97, .97, .97}
\definecolor{messagecolor}{rgb}{0, 0, 0}
\definecolor{warningcolor}{rgb}{1, 0, 1}
\definecolor{errorcolor}{rgb}{1, 0, 0}
\newenvironment{knitrout}{}{} % an empty environment to be redefined in TeX

\usepackage{alltt}
\newcommand{\SweaveOpts}[1]{}  % do not interfere with LaTeX
\newcommand{\SweaveInput}[1]{} % because they are not real TeX commands
\newcommand{\Sexpr}[1]{}       % will only be parsed by R
\newcommand{\xmark}{\ding{55}}%


\usepackage[english]{babel}
\usepackage[utf8]{inputenc}

\usepackage{dsfont}
\usepackage{verbatim}
\usepackage{amsmath}
\usepackage{amsfonts}
\usepackage{amssymb}
\usepackage{bm}
\usepackage{csquotes}
\usepackage{multirow}
\usepackage{longtable}
\usepackage{booktabs}
\usepackage{enumerate}
\usepackage[absolute,overlay]{textpos}
\usepackage{psfrag}
\usepackage{algorithm}
\usepackage{algpseudocode}
\usepackage{eqnarray}
\usepackage{arydshln}
\usepackage{tabularx}
\usepackage{placeins}
\usepackage{tikz}
\usepackage{setspace}
\usepackage{colortbl}
\usepackage{mathtools}
\usepackage{wrapfig}
\usepackage{bm}
\usepackage{amsmath}
\usepackage{pifont}

\usetikzlibrary{shapes,arrows,automata,positioning,calc,chains,trees, shadows}
\tikzset{
  %Define standard arrow tip
  >=stealth',
  %Define style for boxes
  punkt/.style={
    rectangle,
    rounded corners,
    draw=black, very thick,
    text width=6.5em,
    minimum height=2em,
    text centered},
  % Define arrow style
  pil/.style={
    ->,
    thick,
    shorten <=2pt,
    shorten >=2pt,}
}

\usepackage{subfig}

% Defines macros and environments
\usepackage{../../style/lmu-lecture}


\let\code=\texttt
\let\proglang=\textsf

\setkeys{Gin}{width=0.9\textwidth}

\setbeamertemplate{frametitle}{\expandafter\uppercase\expandafter\insertframetitle}


% basic latex stuff
\newcommand{\pkg}[1]{{\fontseries{b}\selectfont #1}} %fontstyle for R packages
\newcommand{\lz}{\vspace{0.5cm}} %vertical space
\newcommand{\dlz}{\vspace{1cm}} %double vertical space
\newcommand{\oneliner}[1] % Oneliner for important statements
{\begin{block}{}\begin{center}\begin{Large}#1\end{Large}\end{center}\end{block}}





% math spaces
\ifdefined\N
\renewcommand{\N}{\mathds{N}} % N, naturals
\else \newcommand{\N}{\mathds{N}} \fi
\newcommand{\Z}{\mathds{Z}} % Z, integers
\newcommand{\Q}{\mathds{Q}} % Q, rationals
\newcommand{\R}{\mathds{R}} % R, reals
\ifdefined\C
  \renewcommand{\C}{\mathds{C}} % C, complex
\else \newcommand{\C}{\mathds{C}} \fi
\newcommand{\continuous}{\mathcal{C}} % C, space of continuous functions
\newcommand{\M}{\mathcal{M}} % machine numbers
\newcommand{\epsm}{\epsilon_m} % maximum error

% counting / finite sets
\newcommand{\setzo}{\{0, 1\}} % set 0, 1
\newcommand{\setmp}{\{-1, +1\}} % set -1, 1
\newcommand{\unitint}{[0, 1]} % unit interval

% basic math stuff
\newcommand{\xt}{\tilde x} % x tilde
\DeclareMathOperator*{\argmax}{arg\,max} % argmax
\DeclareMathOperator*{\argmin}{arg\,min} % argmin
\newcommand{\argminlim}{\mathop{\mathrm{arg\,min}}\limits} % argmax with limits
\newcommand{\argmaxlim}{\mathop{\mathrm{arg\,max}}\limits} % argmin with limits
\newcommand{\sign}{\operatorname{sign}} % sign, signum
\newcommand{\I}{\mathbb{I}} % I, indicator
\newcommand{\order}{\mathcal{O}} % O, order
\newcommand{\bigO}{\mathcal{O}} % Big-O Landau
\newcommand{\littleo}{{o}} % Little-o Landau
\newcommand{\pd}[2]{\frac{\partial{#1}}{\partial #2}} % partial derivative
\newcommand{\floorlr}[1]{\left\lfloor #1 \right\rfloor} % floor
\newcommand{\ceillr}[1]{\left\lceil #1 \right\rceil} % ceiling
\newcommand{\indep}{\perp \!\!\! \perp} % independence symbol

% sums and products
\newcommand{\sumin}{\sum\limits_{i=1}^n} % summation from i=1 to n
\newcommand{\sumim}{\sum\limits_{i=1}^m} % summation from i=1 to m
\newcommand{\sumjn}{\sum\limits_{j=1}^n} % summation from j=1 to p
\newcommand{\sumjp}{\sum\limits_{j=1}^p} % summation from j=1 to p
\newcommand{\sumik}{\sum\limits_{i=1}^k} % summation from i=1 to k
\newcommand{\sumkg}{\sum\limits_{k=1}^g} % summation from k=1 to g
\newcommand{\sumjg}{\sum\limits_{j=1}^g} % summation from j=1 to g
\newcommand{\meanin}{\frac{1}{n} \sum\limits_{i=1}^n} % mean from i=1 to n
\newcommand{\meanim}{\frac{1}{m} \sum\limits_{i=1}^m} % mean from i=1 to n
\newcommand{\meankg}{\frac{1}{g} \sum\limits_{k=1}^g} % mean from k=1 to g
\newcommand{\prodin}{\prod\limits_{i=1}^n} % product from i=1 to n
\newcommand{\prodkg}{\prod\limits_{k=1}^g} % product from k=1 to g
\newcommand{\prodjp}{\prod\limits_{j=1}^p} % product from j=1 to p

% linear algebra
\newcommand{\one}{\boldsymbol{1}} % 1, unitvector
\newcommand{\zero}{\mathbf{0}} % 0-vector
\newcommand{\id}{\boldsymbol{I}} % I, identity
\newcommand{\diag}{\operatorname{diag}} % diag, diagonal
\newcommand{\trace}{\operatorname{tr}} % tr, trace
\newcommand{\spn}{\operatorname{span}} % span
\newcommand{\scp}[2]{\left\langle #1, #2 \right\rangle} % <.,.>, scalarproduct
\newcommand{\mat}[1]{\begin{pmatrix} #1 \end{pmatrix}} % short pmatrix command
\newcommand{\Amat}{\mathbf{A}} % matrix A
\newcommand{\Deltab}{\mathbf{\Delta}} % error term for vectors

% basic probability + stats
\renewcommand{\P}{\mathds{P}} % P, probability
\newcommand{\E}{\mathds{E}} % E, expectation
\newcommand{\var}{\mathsf{Var}} % Var, variance
\newcommand{\cov}{\mathsf{Cov}} % Cov, covariance
\newcommand{\corr}{\mathsf{Corr}} % Corr, correlation
\newcommand{\normal}{\mathcal{N}} % N of the normal distribution
\newcommand{\iid}{\overset{i.i.d}{\sim}} % dist with i.i.d superscript
\newcommand{\distas}[1]{\overset{#1}{\sim}} % ... is distributed as ...

% machine learning
\newcommand{\Xspace}{\mathcal{X}} % X, input space
\newcommand{\Yspace}{\mathcal{Y}} % Y, output space
\newcommand{\Zspace}{\mathcal{Z}} % Z, space of sampled datapoints
\newcommand{\nset}{\{1, \ldots, n\}} % set from 1 to n
\newcommand{\pset}{\{1, \ldots, p\}} % set from 1 to p
\newcommand{\gset}{\{1, \ldots, g\}} % set from 1 to g
\newcommand{\Pxy}{\mathbb{P}_{xy}} % P_xy
\newcommand{\Exy}{\mathbb{E}_{xy}} % E_xy: Expectation over random variables xy
\newcommand{\xv}{\mathbf{x}} % vector x (bold)
\newcommand{\xtil}{\tilde{\mathbf{x}}} % vector x-tilde (bold)
\newcommand{\yv}{\mathbf{y}} % vector y (bold)
\newcommand{\xy}{(\xv, y)} % observation (x, y)
\newcommand{\xvec}{\left(x_1, \ldots, x_p\right)^\top} % (x1, ..., xp)
\newcommand{\Xmat}{\mathbf{X}} % Design matrix
\newcommand{\allDatasets}{\mathds{D}} % The set of all datasets
\newcommand{\allDatasetsn}{\mathds{D}_n}  % The set of all datasets of size n
\newcommand{\D}{\mathcal{D}} % D, data
\newcommand{\Dn}{\D_n} % D_n, data of size n
\newcommand{\Dtrain}{\mathcal{D}_{\text{train}}} % D_train, training set
\newcommand{\Dtest}{\mathcal{D}_{\text{test}}} % D_test, test set
\newcommand{\xyi}[1][i]{\left(\xv^{(#1)}, y^{(#1)}\right)} % (x^i, y^i), i-th observation
\newcommand{\Dset}{\left( \xyi[1], \ldots, \xyi[n]\right)} % {(x1,y1)), ..., (xn,yn)}, data
\newcommand{\defAllDatasetsn}{(\Xspace \times \Yspace)^n} % Def. of the set of all datasets of size n
\newcommand{\defAllDatasets}{\bigcup_{n \in \N}(\Xspace \times \Yspace)^n} % Def. of the set of all datasets
\newcommand{\xdat}{\left\{ \xv^{(1)}, \ldots, \xv^{(n)}\right\}} % {x1, ..., xn}, input data
\newcommand{\ydat}{\left\{ \yv^{(1)}, \ldots, \yv^{(n)}\right\}} % {y1, ..., yn}, input data
\newcommand{\yvec}{\left(y^{(1)}, \hdots, y^{(n)}\right)^\top} % (y1, ..., yn), vector of outcomes
\newcommand{\greekxi}{\xi} % Greek letter xi
\renewcommand{\xi}[1][i]{\xv^{(#1)}} % x^i, i-th observed value of x
\newcommand{\yi}[1][i]{y^{(#1)}} % y^i, i-th observed value of y
\newcommand{\xivec}{\left(x^{(i)}_1, \ldots, x^{(i)}_p\right)^\top} % (x1^i, ..., xp^i), i-th observation vector
\newcommand{\xj}{\xv_j} % x_j, j-th feature
\newcommand{\xjvec}{\left(x^{(1)}_j, \ldots, x^{(n)}_j\right)^\top} % (x^1_j, ..., x^n_j), j-th feature vector
\newcommand{\phiv}{\mathbf{\phi}} % Basis transformation function phi
\newcommand{\phixi}{\mathbf{\phi}^{(i)}} % Basis transformation of xi: phi^i := phi(xi)

%%%%%% ml - models general
\newcommand{\lamv}{\bm{\lambda}} % lambda vector, hyperconfiguration vector
\newcommand{\Lam}{\bm{\Lambda}}	 % Lambda, space of all hpos
% Inducer / Inducing algorithm
\newcommand{\preimageInducer}{\left(\defAllDatasets\right)\times\Lam} % Set of all datasets times the hyperparameter space
\newcommand{\preimageInducerShort}{\allDatasets\times\Lam} % Set of all datasets times the hyperparameter space
% Inducer / Inducing algorithm
\newcommand{\ind}{\mathcal{I}} % Inducer, inducing algorithm, learning algorithm

% continuous prediction function f
\newcommand{\ftrue}{f_{\text{true}}}  % True underlying function (if a statistical model is assumed)
\newcommand{\ftruex}{\ftrue(\xv)} % True underlying function (if a statistical model is assumed)
\newcommand{\fx}{f(\xv)} % f(x), continuous prediction function
\newcommand{\fdomains}{f: \Xspace \rightarrow \R^g} % f with domain and co-domain
\newcommand{\Hspace}{\mathcal{H}} % hypothesis space where f is from
\newcommand{\fbayes}{f^{\ast}} % Bayes-optimal model
\newcommand{\fxbayes}{f^{\ast}(\xv)} % Bayes-optimal model
\newcommand{\fkx}[1][k]{f_{#1}(\xv)} % f_j(x), discriminant component function
\newcommand{\fh}{\hat{f}} % f hat, estimated prediction function
\newcommand{\fxh}{\fh(\xv)} % fhat(x)
\newcommand{\fxt}{f(\xv ~|~ \thetav)} % f(x | theta)
\newcommand{\fxi}{f\left(\xv^{(i)}\right)} % f(x^(i))
\newcommand{\fxih}{\hat{f}\left(\xv^{(i)}\right)} % f(x^(i))
\newcommand{\fxit}{f\left(\xv^{(i)} ~|~ \thetav\right)} % f(x^(i) | theta)
\newcommand{\fhD}{\fh_{\D}} % fhat_D, estimate of f based on D
\newcommand{\fhDtrain}{\fh_{\Dtrain}} % fhat_Dtrain, estimate of f based on D
\newcommand{\fhDnlam}{\fh_{\Dn, \lamv}} %model learned on Dn with hp lambda
\newcommand{\fhDlam}{\fh_{\D, \lamv}} %model learned on D with hp lambda
\newcommand{\fhDnlams}{\fh_{\Dn, \lamv^\ast}} %model learned on Dn with optimal hp lambda
\newcommand{\fhDlams}{\fh_{\D, \lamv^\ast}} %model learned on D with optimal hp lambda

% discrete prediction function h
\newcommand{\hx}{h(\xv)} % h(x), discrete prediction function
\newcommand{\hh}{\hat{h}} % h hat
\newcommand{\hxh}{\hat{h}(\xv)} % hhat(x)
\newcommand{\hxt}{h(\xv | \thetav)} % h(x | theta)
\newcommand{\hxi}{h\left(\xi\right)} % h(x^(i))
\newcommand{\hxit}{h\left(\xi ~|~ \thetav\right)} % h(x^(i) | theta)
\newcommand{\hbayes}{h^{\ast}} % Bayes-optimal classification model
\newcommand{\hxbayes}{h^{\ast}(\xv)} % Bayes-optimal classification model

% yhat
\newcommand{\yh}{\hat{y}} % yhat for prediction of target
\newcommand{\yih}{\hat{y}^{(i)}} % yhat^(i) for prediction of ith targiet
\newcommand{\resi}{\yi- \yih}

% theta
\newcommand{\thetah}{\hat{\theta}} % theta hat
\newcommand{\thetav}{\bm{\theta}} % theta vector
\newcommand{\thetavh}{\bm{\hat\theta}} % theta vector hat
\newcommand{\thetat}[1][t]{\thetav^{[#1]}} % theta^[t] in optimization
\newcommand{\thetatn}[1][t]{\thetav^{[#1 +1]}} % theta^[t+1] in optimization
\newcommand{\thetahDnlam}{\thetavh_{\Dn, \lamv}} %theta learned on Dn with hp lambda
\newcommand{\thetahDlam}{\thetavh_{\D, \lamv}} %theta learned on D with hp lambda
\newcommand{\mint}{\min_{\thetav \in \Theta}} % min problem theta
\newcommand{\argmint}{\argmin_{\thetav \in \Theta}} % argmin theta

% densities + probabilities
% pdf of x
\newcommand{\pdf}{p} % p
\newcommand{\pdfx}{p(\xv)} % p(x)
\newcommand{\pixt}{\pi(\xv~|~ \thetav)} % pi(x|theta), pdf of x given theta
\newcommand{\pixit}[1][i]{\pi\left(\xi[#1] ~|~ \thetav\right)} % pi(x^i|theta), pdf of x given theta
\newcommand{\pixii}[1][i]{\pi\left(\xi[#1]\right)} % pi(x^i), pdf of i-th x

% pdf of (x, y)
\newcommand{\pdfxy}{p(\xv,y)} % p(x, y)
\newcommand{\pdfxyt}{p(\xv, y ~|~ \thetav)} % p(x, y | theta)
\newcommand{\pdfxyit}{p\left(\xi, \yi ~|~ \thetav\right)} % p(x^(i), y^(i) | theta)

% pdf of x given y
\newcommand{\pdfxyk}[1][k]{p(\xv | y= #1)} % p(x | y = k)
\newcommand{\lpdfxyk}[1][k]{\log p(\xv | y= #1)} % log p(x | y = k)
\newcommand{\pdfxiyk}[1][k]{p\left(\xi | y= #1 \right)} % p(x^i | y = k)

% prior probabilities
\newcommand{\pik}[1][k]{\pi_{#1}} % pi_k, prior
\newcommand{\lpik}[1][k]{\log \pi_{#1}} % log pi_k, log of the prior
\newcommand{\pit}{\pi(\thetav)} % Prior probability of parameter theta

% posterior probabilities
\newcommand{\post}{\P(y = 1 ~|~ \xv)} % P(y = 1 | x), post. prob for y=1
\newcommand{\postk}[1][k]{\P(y = #1 ~|~ \xv)} % P(y = k | y), post. prob for y=k
\newcommand{\pidomains}{\pi: \Xspace \rightarrow \unitint} % pi with domain and co-domain
\newcommand{\pibayes}{\pi^{\ast}} % Bayes-optimal classification model
\newcommand{\pixbayes}{\pi^{\ast}(\xv)} % Bayes-optimal classification model
\newcommand{\pix}{\pi(\xv)} % pi(x), P(y = 1 | x)
\newcommand{\piv}{\bm{\pi}} % pi, bold, as vector
\newcommand{\pikx}[1][k]{\pi_{#1}(\xv)} % pi_k(x), P(y = k | x)
\newcommand{\pikxt}[1][k]{\pi_{#1}(\xv ~|~ \thetav)} % pi_k(x | theta), P(y = k | x, theta)
\newcommand{\pixh}{\hat \pi(\xv)} % pi(x) hat, P(y = 1 | x) hat
\newcommand{\pikxh}[1][k]{\hat \pi_{#1}(\xv)} % pi_k(x) hat, P(y = k | x) hat
\newcommand{\pixih}{\hat \pi(\xi)} % pi(x^(i)) with hat
\newcommand{\pikxih}[1][k]{\hat \pi_{#1}(\xi)} % pi_k(x^(i)) with hat
\newcommand{\pdfygxt}{p(y ~|~\xv, \thetav)} % p(y | x, theta)
\newcommand{\pdfyigxit}{p\left(\yi ~|~\xi, \thetav\right)} % p(y^i |x^i, theta)
\newcommand{\lpdfygxt}{\log \pdfygxt } % log p(y | x, theta)
\newcommand{\lpdfyigxit}{\log \pdfyigxit} % log p(y^i |x^i, theta)

% probababilistic
\newcommand{\bayesrulek}[1][k]{\frac{\P(\xv | y= #1) \P(y= #1)}{\P(\xv)}} % Bayes rule
\newcommand{\muk}{\bm{\mu_k}} % mean vector of class-k Gaussian (discr analysis)

% residual and margin
\newcommand{\eps}{\epsilon} % residual, stochastic
\newcommand{\epsv}{\bm{\epsilon}} % residual, stochastic, as vector
\newcommand{\epsi}{\epsilon^{(i)}} % epsilon^i, residual, stochastic
\newcommand{\epsh}{\hat{\epsilon}} % residual, estimated
\newcommand{\epsvh}{\hat{\epsv}} % residual, estimated, vector
\newcommand{\yf}{y \fx} % y f(x), margin
\newcommand{\yfi}{\yi \fxi} % y^i f(x^i), margin
\newcommand{\Sigmah}{\hat \Sigma} % estimated covariance matrix
\newcommand{\Sigmahj}{\hat \Sigma_j} % estimated covariance matrix for the j-th class

% ml - loss, risk, likelihood
\newcommand{\Lyf}{L\left(y, f\right)} % L(y, f), loss function
\newcommand{\Lypi}{L\left(y, \pi\right)} % L(y, pi), loss function
\newcommand{\Lxy}{L\left(y, \fx\right)} % L(y, f(x)), loss function
\newcommand{\Lxyi}{L\left(\yi, \fxi\right)} % loss of observation
\newcommand{\Lxyt}{L\left(y, \fxt\right)} % loss with f parameterized
\newcommand{\Lxyit}{L\left(\yi, \fxit\right)} % loss of observation with f parameterized
\newcommand{\Lxym}{L\left(\yi, f\left(\bm{\tilde{x}}^{(i)} ~|~ \thetav\right)\right)} % loss of observation with f parameterized
\newcommand{\Lpixy}{L\left(y, \pix\right)} % loss in classification
\newcommand{\Lpiv}{L\left(y, \piv\right)} % loss in classification
\newcommand{\Lpixyi}{L\left(\yi, \pixii\right)} % loss of observation in classification
\newcommand{\Lpixyt}{L\left(y, \pixt\right)} % loss with pi parameterized
\newcommand{\Lpixyit}{L\left(\yi, \pixit\right)} % loss of observation with pi parameterized
\newcommand{\Lhxy}{L\left(y, \hx\right)} % L(y, h(x)), loss function on discrete classes
\newcommand{\Lr}{L\left(r\right)} % L(r), loss defined on residual (reg) / margin (classif)
\newcommand{\lone}{|y - \fx|} % L1 loss
\newcommand{\ltwo}{\left(y - \fx\right)^2} % L2 loss
\newcommand{\lbernoullimp}{\ln(1 + \exp(-y \cdot \fx))} % Bernoulli loss for -1, +1 encoding
\newcommand{\lbernoullizo}{- y \cdot \fx + \log(1 + \exp(\fx))} % Bernoulli loss for 0, 1 encoding
\newcommand{\lcrossent}{- y \log \left(\pix\right) - (1 - y) \log \left(1 - \pix\right)} % cross-entropy loss
\newcommand{\lbrier}{\left(\pix - y \right)^2} % Brier score
\newcommand{\risk}{\mathcal{R}} % R, risk
\newcommand{\riskbayes}{\mathcal{R}^\ast}
\newcommand{\riskf}{\risk(f)} % R(f), risk
\newcommand{\riskdef}{\E_{y|\xv}\left(\Lxy \right)} % risk def (expected loss)
\newcommand{\riskt}{\mathcal{R}(\thetav)} % R(theta), risk
\newcommand{\riske}{\mathcal{R}_{\text{emp}}} % R_emp, empirical risk w/o factor 1 / n
\newcommand{\riskeb}{\bar{\mathcal{R}}_{\text{emp}}} % R_emp, empirical risk w/ factor 1 / n
\newcommand{\riskef}{\riske(f)} % R_emp(f)
\newcommand{\risket}{\mathcal{R}_{\text{emp}}(\thetav)} % R_emp(theta)
\newcommand{\riskr}{\mathcal{R}_{\text{reg}}} % R_reg, regularized risk
\newcommand{\riskrt}{\mathcal{R}_{\text{reg}}(\thetav)} % R_reg(theta)
\newcommand{\riskrf}{\riskr(f)} % R_reg(f)
\newcommand{\riskrth}{\hat{\mathcal{R}}_{\text{reg}}(\thetav)} % hat R_reg(theta)
\newcommand{\risketh}{\hat{\mathcal{R}}_{\text{emp}}(\thetav)} % hat R_emp(theta)
\newcommand{\LL}{\mathcal{L}} % L, likelihood
\newcommand{\LLt}{\mathcal{L}(\thetav)} % L(theta), likelihood
\newcommand{\LLtx}{\mathcal{L}(\thetav | \xv)} % L(theta|x), likelihood
\newcommand{\logl}{\ell} % l, log-likelihood
\newcommand{\loglt}{\logl(\thetav)} % l(theta), log-likelihood
\newcommand{\logltx}{\logl(\thetav | \xv)} % l(theta|x), log-likelihood
\newcommand{\errtrain}{\text{err}_{\text{train}}} % training error
\newcommand{\errtest}{\text{err}_{\text{test}}} % test error
\newcommand{\errexp}{\overline{\text{err}_{\text{test}}}} % avg training error

% lm
\newcommand{\thx}{\thetav^\top \xv} % linear model
\newcommand{\olsest}{(\Xmat^\top \Xmat)^{-1} \Xmat^\top \yv} % OLS estimator in LM

% ml - svms
\newcommand{\sv}{\operatorname{SV}}                                         % supportvectors
\newcommand{\HS}{\mathcal{H}}                                               % H, hilbertspace
\let\myxi\xi																% xi, slack variable (SVM)


\newcommand{\titlefigure}{figure/svm_geometry}
\newcommand{\learninggoals}{
  \item Know how to derive the SVM dual problem
}

\title{Introduction to Machine Learning}
\date{}

\begin{document}

\lecturechapter{Hard-Margin SVM Dual}
\lecture{Introduction to Machine Learning}

\sloppy 



%\begin{vbframe}{Constrained Optimization}

%\small
%\textbf{Remark:} \emph{In this section, $f: \R^n \to \R$ and $h: \R^n \to \R^l$ are arbitrary differentiable functions and not specifically prediction functions.}
%
%\vspace*{0.1cm}
%
%\normalsize
%A general constrained minimization problem has the form
%
%\vspace*{-0.5cm}
%
%  \begin{align*}
%     \min\limits_{\thetab} &\, f(\thetab)  \\
%    \text{s.t. } & \quad g_i(\thetab)  \le 0 \quad\forall\, i \in \{1, ..., k \} &\quad(\text{inequality constraints}) \\
%    h_j(\thetab) & = 0 \quad\forall\, j \in \{1, ..., l \} &\quad(\text{equality constraints})
%  \end{align*}
%
%This is called the \textbf{primal form}. \\
%Let $\fh_P$ be the minimal value of $f(\thetab)$ satisfying all of the constraints.
%
%\framebreak
%
%The corresponding \textbf{Lagrangian} is
%
%$$
%L(\thetab, \alpha, \beta) = f(\thetab) + \sum_{i=1}^k \alpha_i g_i(\thetab) + \sum_{j=1}^l \beta_j h_j(\thetab)
%$$

%with the \textbf{Lagrange Multipliers} $\alpha_i, \beta_j \ge 0$.

%\lz 

%Minimizing the Lagrangian is minimizing $f(\thetab)$ under the constraints:\\ 
%Since $\frac{\partial L(\thetab, \alpha, \beta)}{\partial \beta_j}= h_j(\thetab)$,
%setting the partial derivative of $L(\thetab, \beta)$ w.r.t. $\beta_j$ to 0 just restates the constraints.\\
%Since $\frac{\partial L(\thetab, \alpha, \beta)}{\partial \thetab} = \nabla f(\thetab) + \sum_{i=1}^k \alpha_i \nabla g_i(\thetab) + \sum_{j=1}^l \beta_j \nabla h_j(\thetab)$, setting the partial derivative of $L(\%theta, \beta)$ w.r.t. $\thetab$ to 0 also aligns the gradients of the objective and constraint functions (see following slides).\\


%\end{vbframe}
%
%\begin{vbframe}{Intuition behind the Lagrangian function}
%
%% https://www.khanacademy.org/math/multivariable-calculus/applications-of-multivariable-derivatives/constrained-optimization/a/lagrange-multipliers-single-constraint
%
%Assume a very simple optimization problem

%\vspace*{-0.5cm}
%
%  \begin{eqnarray*}
%    \min\limits_{\thetab_1, \thetab_2} &  \thetab_1 + \thetab_2 & \\
%    \text{s.t.} & \thetab_1^2 + \thetab_2^2 & = 1
%  \end{eqnarray*}
%
%\lz
%
%The Lagrangian function is
%$$
%L(\thetab, \beta) = f(\thetab) + \beta h(\thetab) = (\thetab_1 + \thetab_2) + \beta(\thetab_1^2 + \thetab_2^2 - 1)
%$$
%
%You can think of the optimization problem $L(\thetab, \beta) \to \min$ as moving the (linear) contours of the objective forward and backward until you find a point $(\thetab_1, \thetab_2)$ that fulfills the %constraint \emph{and} minimizes the objective function or as moving around the feasible region given by the constraint until you are at the minimum of the objective function.
% 
% \center
%   \includegraphics{figure_man/optimization/constraints_violated.pdf}
% 
% \framebreak
% 
%   \includegraphics{figure_man/optimization/constraints_satisfied.pdf}
% 
% \framebreak
% 
%   \includegraphics{figure_man/optimization/constraints_opt.pdf}

%<<constraints_opt-3d, fig.height = 8, fig.width = 8, out.height = '.8\\textheight', out.width = '.8\\textheight'>>=
%g = seq(-1, 1, l = 200)
%xc1 = c(g, rev(g))
%xc2 = c(sqrt((1 - g^2)), -sqrt((1 - g^2)))
%circle = data.frame(x1 = xc1, x2 = xc2)
%x1 = x2 = seq(-2.5, 1.5, l = 250)
%circle = data.frame(x1 = xc1, x2 = xc2)
%x1 = x2 = seq(-1.5, 2.5, l = 250)
% p <- persp(x1, x2, outer(x1, x2, `+`), col = rgb(0,0,0,.5), border = NA, shade = TRUE,
%   xlab = "theta_1", ylab = "theta_2", zlab = "f(theta)", ticktype = "detailed", 
%   theta = 20, phi = 30)
% lines(trans3d(x = xc1, y = xc2, z = xc1+xc2, p), col = "red", lwd = 2)
% @
% 
% <<constraints_opt, fig.height = 8, fig.width = 8, out.height = '.8\\textheight', out.width = '.8\\textheight'>>=
% image(x1, x2, outer(x1, x2, `+`), asp = 1, col = viridis::viridis(45), 
%   xlab = expression(theta[1]), ylab = expression(theta[2]), useRaster = TRUE)
% contours <- c(-4, -3, round(-sqrt(2), 2), 0, round(sqrt(2), 2), 2, 3)
% contour(x1, x2, outer(x1, x2, `+`), add = TRUE, levels = contours,
%   labels = paste0("theta_1 + theta_2 =", contours), labcex = 1, col = "white")
% lines(circle, col = "red", lwd = 2)
% text(x = -1.3, y = 0.8, labels = expression(theta[1]^2 + theta[2]^2 == 1), col = "red", cex = 1.3)
% points(x = -sqrt(2)/2, y = -sqrt(2)/2, col = "white", pch = 19)
% f_gradient = data.frame(x0 = c(1, -2, -sqrt(2)/2), y0 = c(2, -1, -sqrt(2)/2))
% with(f_gradient, arrows(x0, y0, x1 = x0+.2, y1 = y0+.2,  col= "white", length = .1))
% g_gradient = data.frame(x0 = c(0, -1, 0, -sqrt(2)/2), y0 = c(-1, 0, 1, -sqrt(2)/2))
% with(g_gradient, arrows(x0, y0, x1 = x0 + .2*x0, y1 = y0 + .2*y0,  col= "red", length = .1))
% @
% 
% 
% 
% 
% \framebreak
%  
% \flushleft
% At the optima, the contour lines of the objective function $f(\thetab)$ are tangential to the constraint $h(\thetab)$.
% 
% \lz
% 
% This means that the gradients of $f(\thetab)$ and $h(\thetab)$ are parallel (since the gradient is orthogonal to the contour line), so $\nabla f(\thetab) \propto \nabla h(\thetab)$, so:
% 
% \begin{align*}
% \exists \beta:  \nabla f(\thetab) &= - \beta \nabla h(\thetab) \text{ for optimal $\thetab$}\\
% \intertext{which is equivalent to}
%   \nabla f(\thetab) + \beta \nabla h(\thetab) &= 0 \\
%   \text{i.e., }\qquad \frac{\partial}{\partial \thetab} L(\thetab, \beta) &= 0,
% \end{align*}
% so finding a minimum of the Lagrangian solves the constrained optimization problem.\\
% This idea is extended to more complex objective functions and constraints.
% 
% \end{vbframe}


% \begin{vbframe}{Lagrangian Duality}
% 
% Using the Lagrangian, we can write the primal problem equivalently as
% $$
% \min\limits_{\thetab} \max\limits_{\alpha, \beta} L(\thetab, \alpha, \beta).
% $$
% For any given  $\thetab$:\\
% If $\thetab$ satisfies the constraints, $L(\thetab, \alpha, \beta) \equiv f(\thetab)$ so $\alpha, \beta$ are irrelevant.\linebreak 
% If $\thetab$ doesn't, we want $L(\thetab, \alpha, \beta) \to \infty$, because the constraints \emph{need} to be satisfied.
% 
% \framebreak
% 
% The primal problem is a \textbf{really} hard problem for complicated $f(), g(), h()$.  
% We obtain the \textbf{dual} optimization problem by switching $\max$ and $\min$, i. e.
% 
% $$
% \max\limits_{\alpha, \beta} \min\limits_{\thetab} L(\thetab, \alpha, \beta).
% $$
% Since $L(\thetab, \alpha, \beta)$ is a simple sum over $\alpha, \beta$ for fixed $\thetab$, the
% optimzation here is much easier.
% 
% Let $\fh_D$ be the optimal value of the dual problem.
% It always provides a lower bound to the optimal value of the primal problem
% 
% $$\fh_D \le \fh_P \quad \text{(weak duality)}$$
% 
% and for convex optimization problems (under regularity) we have
% 
% $$\fh_D = \fh_P \quad \text{(strong duality)}$$
% 
% % Slater's condition
% 
% \framebreak
% 
% In case of strong duality, a necessary and sufficient condition for a solution to the primal and dual problem $\thetabh, \hat \alpha, \hat \beta$ are the \textbf{Karush-Kuhn-Tucker-Conditions}
% 
% \begin{align*}
% \left.\fp{L(\thetab, \alpha, \beta)}{\thetab}\right|_{(\thetabh, \hat\alpha, \hat\beta)} &= 0  &\text{(Stationarity)}\\
% g_i(\thetabh) &\le 0 \quad \forall ~ i \in \{1, ..., k \} &\text{(Primal feasibility I)}\\
% h_j(\thetabh) &= 0 \quad \forall ~ j \in \{1, ..., l \} & \text{(Primal feasibility II)}\\
% \hat \alpha_i &\ge 0 \quad \forall ~ i \in \{1, ..., k \} &\text{(Dual feasibility)}\\
% \hat \alpha_i g_i(\thetabh) &= 0 \quad \forall ~ i \in \{1, ..., k \} &\text{(Complementarity)}
% \end{align*}
% 
% Particularly for SVMs, the KKT conditions are necessary and sufficient for a solution.
% 
% \end{vbframe}

\begin{vbframe}{Hard Margin SVM Dual}

We before derived the primal quadratic program for the hard margin SVM. We could directly solve this, but traditionally the SVM is solved in the dual and this has some advantages. In any case, many algorithms and derivations are based on it, so we need to know it.
  \begin{eqnarray*}
  & \min\limits_{\thetab, \theta_0} \quad & \frac{1}{2} \|\thetab\|^2 \\
  & \text{s.t.} & \,\,\yi  \left( \scp{\thetab}{\xi} + \theta_0 \right) \geq 1 \quad \forall\, i \in \nset.
\end{eqnarray*}


% \end{vbframe}


% \begin{vbframe}{Duality in SVM}

% \begin{footnotesize}
% \textbf{Remark:} For a recap on constrained optimization and duality see CIM1 - Statistical Computing. 
% \end{footnotesize}

% \lz 

The Lagrange function of the SVM optimization problem is

\vspace*{-.5cm}

\small
\begin{eqnarray*}
&L(\thetab, \theta_0, \alphav) = & \frac{1}{2}\|\thetab\|^2  -  \sum_{i=1}^n \alpha_i \left[\yi  \left( \scp{\thetab}{\xi} + \theta_0 \right) - 1\right]\\
 & \text{s.t.} & \,\, \alpha_i \ge 0 \quad \forall\, i \in \nset.
\end{eqnarray*}
\small

The \textbf{dual} form of this problem is
$$\max\limits_{\alpha} \min\limits_{\thetab, \theta_0}  L(\thetab, \theta_0,\alphav).$$

\framebreak 

Notice how the (p+1) decision variables $(\thetab,\theta_0)$ have become $n$ decisions variables $\alphav$, as constraints turned into variables and vice versa.
Now every data point has an associated non-negative weight.
\begin{eqnarray*}
&L(\thetab, \theta_0, \alphav) = & \frac{1}{2}\|\thetab\|^2  -  \sum_{i=1}^n \alpha_i \left[\yi  \left( \scp{\thetab}{\xi} + \theta_0 \right) - 1\right]\\
 & \text{s.t.} & \,\, \alpha_i \ge 0 \quad \forall\, i \in \nset.
\end{eqnarray*}
We find the stationary point of $L(\thetab, \theta_0,\alphav)$ w.r.t. $\thetab, \theta_0$ and obtain
\begin{eqnarray*}
    \thetab & = & \sum_{i=1}^n \alpha_i \yi \xi, \\
    0 & = & \sum_{i=1}^n \alpha_i \yi \quad \forall\, i \in \nset.\\
\end{eqnarray*}


\framebreak 

By inserting these expressions 
% 1/2 \sum_{i,j}\alpha_i\alpha_j y_i y_j <x_i, x_j> - \sum_{i,j}\alpha_i\alpha_j y_i y_j <x_i, x_j> - \sum_i alpha_i y_i \thetab0 + \sum_i \alpha_i
\& simplifying we obtain the dual problem

\vspace*{-0.5cm}
\begin{eqnarray*}
    & \max\limits_{\alphav \in \R^n} & \sum_{i=1}^n \alpha_i - \frac{1}{2}\sum_{i=1}^n\sum_{j=1}^n\alpha_i\alpha_j\yi y^{(j)} \scp{\xi}{\xv^{(j)}} \\
    & \text{s.t.} & \sum_{i=1}^n \alpha_i \yi = 0, \\
    & \quad & \alpha_i \ge 0~\forall i \in \nset,
\end{eqnarray*}

or, equivalently, in matrix notation:

\vspace*{-.5cm}
\begin{eqnarray*}
  & \max\limits_{\alphav \in \R^n} & \one^T \alphav - \frac{1}{2} \alphav^T \diag(\yv)\bm{K} \diag(\yv) \alphav \\
  & \text{s.t.} & \alphav^T \yv = 0, \\
  & \quad & \alphav \geq 0,
\end{eqnarray*}

with $\bm{K}:= \Xmat \Xmat^T$.

\framebreak

If $(\thetab, \theta_0, \alphav)$ fulfills the KKT conditions (stationarity, primal/dual feasibility, complementary slackness), it solves both the primal and dual problem (strong duality). 


Under these conditions, and if we solve the dual problem and obtain $\alphavh$, we know that $\thetab$ is a linear combination of our data points:
  
$$
   \thetah = \sumin \alphah_i \yi \xi 
$$

Complementary slackness means:

$$
\alphah_i \left[\yi  \left( \scp{\thetab}{\xi} + \theta_0 \right) - 1\right] = 0 \quad \forall ~ i \in \{1, ..., n \}.
$$

\framebreak

$$
   \thetah = \sumin \alphah_i \yi \xi 
$$
$$
\alphah_i \left[\yi  \left( \scp{\thetab}{\xi} + \theta_0 \right) - 1\right] = 0 \quad \forall ~ i \in \{1, ..., n \}.
$$

\begin{itemize}
  \item So either $\alphah_i = 0$, and is not active in the linear combination,
    or $\alphah_i > 0$, then $\yi \left( \scp{\thetab}{\xi} + \theta_0 \right) = 1$, and $(\xi, \yi)$ has minimal margin and is a support vector!
  \item We see that we can directly extract the support vectors from the dual variables and the $\thetab$ solution only depends on them.
  \item We can reconstruct the bias term $\theta_0$ from any support vector:
  $$
  \theta_0 = \yi - \scp{\thetab}{\xi}.
  $$
\end{itemize}

\end{vbframe}


\endlecture
\end{document}


